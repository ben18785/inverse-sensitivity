\documentclass[10pt,letterpaper]{article}
\usepackage[top=0.85in,left=2.75in,footskip=0.75in]{geometry}

% amsmath and amssymb packages, useful for mathematical formulas and symbols
\usepackage{amsmath,amssymb}
\DeclareMathOperator*{\argmax}{arg\,max}

\usepackage{graphicx}
\usepackage{booktabs}

% Use adjustwidth environment to exceed column width (see example table in text)
\usepackage{changepage}
\usepackage{tabularx}

% Use Unicode characters when possible
\usepackage{inputenc}

% textcomp package and marvosym package for additional characters
\usepackage{textcomp,marvosym}

% cite package, to clean up citations in the main text. Do not remove.
\usepackage{cite}

% Use nameref to cite supporting information files (see Supporting Information section for more info)
\usepackage{nameref,hyperref}

% line numbers
\usepackage[right]{lineno}

\usepackage{tikz}

% ligatures disabled
\usepackage{microtype}
\DisableLigatures[f]{encoding = *, family = * }


% array package and thick rules for tables
\usepackage{array}

\usepackage{algorithm}
\usepackage{algpseudocode}

\usepackage{float}
\usepackage{timestamp}
\usepackage{xfrac}
\usepackage{mathtools}

%% Include all macros below

\newcommand{\lorem}{{\bf LOREM}}
\newcommand{\ipsum}{{\bf IPSUM}}

%% END MACROS SECTION
\newcommand{\R}{\mathbb{R}}

\begin{document}
\vspace*{0.2in}

%% Title must be 250 characters or less.
%\begin{flushleft}
%{\Large
%\textbf\newline{A Monte Carlo method to estimate cell population heterogeneity}
%}
%\newline
%\\
%Ben Lambert\textsuperscript{1,2}*,
%David J. Gavaghan\textsuperscript{3},
%Simon Tavener\textsuperscript{4}.
%\\
%\bigskip
%\textbf{1} Department of Zoology, University of Oxford, Oxford, Oxfordshire, U.K.
%\\
%\textbf{2} MRC Centre for Global Infectious Disease Analysis, School of Public Health, Imperial College London, London W2 1PG, UK.
%\\
%\textbf{3} Department of Computer Science, University of Oxford, Oxford, U.K.
%\\
%\textbf{4} Department of Mathematics, Colorado State University, Fort Collins, Colorado, U.S.A.
%\\
%\bigskip
%
%% Use the asterisk to denote corresponding authorship and provide email address in note below.
%*ben.c.lambert@gmail.com.
%
%\end{flushleft}

\hfill Revision date \& time: \timestamp
\bigskip


% Please keep the abstract below 300 words
%%%%%%%%%%%%%%%%%%%%%%%%%%%%%%%%%%%%%%%%%%%%%%%%%%%%%%%%%%%%%%%%%%%%%%%%%%%%%%%%%%%%%%%%%%%%%%%%%%%%%%%%%%%%%%%%%%%%%%%%                                                                                                                  %                                                                                                                      %
%       ABSTRACT                                                                                                       %
%                                                                                                                      %
%%%%%%%%%%%%%%%%%%%%%%%%%%%%%%%%%%%%%%%%%%%%%%%%%%%%%%%%%%%%%%%%%%%%%%%%%%%%%%%%%%%%%%%%%%%%%%%%%%%%%%%%%%%%%%%%%%%%%%%%

% The Abstract of the paper should be succinct; it must not exceed 300 words. Authors should mention the techniques used without going into methodological detail and should summarize the most important results.

% While the Abstract is conceptually divided into three sections (Background, Methodology/Principal Findings, and Conclusions/Significance), do not apply these distinct headings to the Abstract within the article file.

% Do not include any citations. Avoid specialist abbreviations.


\section{Introduction}

We consider a simple dynamical system model based on Michaelis-Menten reaction kinetics \cite{tu2019single} which seeks to explain a subset of the single-cell RNA-sequence data of  embryonic stem cell development reported by \cite{chu2016single}.

\section{Mathematical model}

%\subsection{Initial value problem}
Let $C=$ [CDH1], $Z=$ [ZEB1] and $K=$ [KLF8] represent the time-dependent concentrations of three genes that have been recognized as key actors in embryonic stem cell development. Let
\begin{eqnarray}
\frac{dC}{d\tau} &=& \frac{k_1}{k_2 + Z^2} + \frac{k_3}{k_4 + K^2} - d_1 C  \label{eq:de_one} \\
\frac{dZ}{d\tau} &=& \frac{a k_5 K^2}{k_6 + K^2} - d_2 Z \label{eq:de_two} \\
\frac{dK}{d\tau} &=& \frac{r k_7}{k_8 + C^2} - d_3 K \label{eq:de_three}
\end{eqnarray}
subject to initial conditions
\begin{equation}
C(0) = C_0, \quad Z(0) = Z_0, \quad K(0) = K_0,
\end{equation}
where $k_1, k_3, k_5, k_7$ are rates of reaction, $k_2, k_4, k_6, k_8$ are Michaelis parameters, $d_1, d_2, d_3$ are decay rates and $a$ and $r$ are dimensionless constants.

In order to dimensionalize this system of equations we introduce dimensionless variables and parameters.

\section{Nondimensionalization}

\subsection{Units of dimensional parameters}

$C, K$ and $Z$ are concentrations with units $L^{-3}$ and therefore all the LHS terms e.g., $\frac{dC}{d\tau}$ have units $L^{-3} T^{-1}$. To be consistent, therefore

\begin{itemize}
  \item $d_1, d_2, d_3$ have units $T^{-1}$
  \item $k_2, k_4, k_6, k_8$ have units of $L^{-6}$ %so that the denominators have consistent units
  \item $k_1, k_3$ have units $L^{-9} T^{-1}$
  \item $k_7$ has units $L^{-9} T^{-1}$ since $r$ is dimensionless
  \item $k_5$ has units $L^{-3}T^{-1}$  since $a$ is dimensionless
\end{itemize}


\subsection{Nondimensional variables}
The nondimensional variables are
\begin{equation}\label{eq:variables}
y_1 = \left(\frac{k_2 d_1}{k_1}\right) C, \qquad
y_2 = \left(\frac{k_6 d_1}{k_4 k_5}\right) Z, \qquad
y_3 = \left(\frac{1}{\sqrt{k_4}}\right) K.
\end{equation}
We note that
\begin{itemize}
  \item $\dfrac{k_2 d_1}{k_1}$ has units $L^3$ therefore $y_1$ is dimensionless,
  \item $\dfrac{k_6 d_1}{k_4 k_5}$ has units $L^3$ therefore $y_2$ is dimensionless,
  \item $\dfrac{1}{\sqrt{k_4}}$ has units $L^3$ therefore $y_3$ is dimensionless.
\end{itemize}


\subsection{Nondimensional groups (parameters)}

The nondimensional groups (parameters) are ...
\begin{equation}\label{eq:parameters}
\left.
\begin{gathered}\begin{aligned}
p_1 &= \frac{k_4^2 k_5^2}{k_2 k_6^2 d_1^2},  \\
p_3 &= \frac{k_4}{k_6}, \\
p_5 &= \frac{k_2^2 k_7 d_1}{k_1^2 \sqrt{k_4}}, \\
p_7 &= \frac{d_3}{d_1}.
\end{aligned}\end{gathered}
\qquad \qquad
\begin{gathered}\begin{aligned}
p_2 &= \frac{k_2 k_3}{k_1 k_4}, \\
p_4 &= \frac{d_2}{d_1}, \\
p_6 &= \frac{k_2^2 k_8 d_1^2}{k_1^2}, \\
& \\
\end{aligned}\end{gathered}
\qquad \right \}
\end{equation}

Parameters $p_3, p_4$ and $p_7$ are trivially dimensionless  since they are ratios of dimensional parameter with equal units. The remaining parameters are not so obviously dimensionless. However,
\begin{itemize}
  \item $p_1$ has units $\dfrac{L^{-12}\cdot L^{-6}T^{-2}}  {L^{-6} \cdot L^{-12} \cdot T^{-2}} = 1$,
  \item $p_2$ has units $\dfrac{L^{-6} \cdot L^{-9}T^{-1}}  {L^{-9} T^{-1} \cdot L^{-6}} = 1$,
  \item $p_5$ has units $\dfrac{L^{-12}\cdot L^{-9}T^{-1} \cdot T^{-1} }  {L^{-18} T^{-2} \cdot L^{-3}} = 1$,
  \item $p_6$ has units $\dfrac{L^{-12}\cdot L^{-6} \cdot T^{-2}}  {L^{-18}\cdot T^{-2}} = 1$.
\end{itemize}


\subsection{Nondimensionalizing governing equations}

The terms by which to multiply the LHS and RHS of equations \ref{eq:de_one}, \ref{eq:de_two} and \ref{eq:de_three} are most easily determined by considering the decay term. The time scale is $\dfrac{1}{d_1}$.

\subsubsection{Equation \eqref{eq:de_one}}
Multiply \eqref{eq:de_one} by $\dfrac{k_2}{k_1}$.
\begin{equation*} 
\begin{gathered}\begin{aligned}
{\rm LHS} &= \frac{1}{d_1} \frac{k_2 d_1}{k_1} \frac{dC}{d\tau}
           = \frac{1}{d_1} \frac{dy_1}{d\tau}
           = \frac{dy_1}{dt} \\
{\rm RHS(1)} &= \frac{1}{1 + Z^2/k_2}
              = \frac{1}{1 + (k_4 k_5 y_2 / k_6 d_1)^2/k_2}
              = \frac{1}{1 + p_1 y_2^2} \\
{\rm RHS(2)} &= \frac{k_2 k_3}{k_1 k_4} \left( \frac{1}{1 + k_4 y_3^2/k_4} \right)
              = \frac{p_2}{1 + y_3^2}  \\
{\rm RHS(3)} &= \frac{k_2 d_1}{k_1} C = y_1
\end{aligned}\end{gathered}
\end{equation*}

\subsubsection{Equation \eqref{eq:de_two}}
Multiply \eqref{eq:de_two} by $\dfrac{k_6}{k_4 k_5}$.
\begin{equation*}
\begin{gathered}\begin{aligned}
{\rm LHS} &= \frac{1}{d_1} \frac{k_6 d_1}{k_4 k_5} \frac{dZ}{d\tau}
           = \frac{1}{d_1} \frac{dy_2}{d\tau}
           = \frac{dy_2}{dt} \\
{\rm RHS(1)} &= \frac{k_6}{k_4} \left( \frac{K^2}{k_6(1 + K^2/k_6)} \right)
              = \frac{K^2/k_4}{(1 + (k_4/k_6)(K^2/k_4))}
              = \frac{y_3^2}{1 + p_3 y_3^2} \\
{\rm RHS(2)} &= \frac{d_2}{d_1}\left(\frac{k_6 d_1}{k_4 k_5}\right) Z
              = p_4 y_2
\end{aligned}\end{gathered}
\end{equation*}


\subsubsection{Equation \eqref{eq:de_three}}
Multiply \eqref{eq:de_three} multiply by $\dfrac{1}{d_1 \sqrt{k_4}}$.
\begin{equation*} 
\begin{gathered}\begin{aligned}
{\rm LHS} &= \frac{1}{d_1 \sqrt{k_4}} \frac{dK}{d\tau}
           = \frac{1}{d_1} \frac{dy_3}{d\tau}
           = \frac{dy_3}{dt} \\
{\rm RHS(1)} &= \left( \frac{k_2^2 k_7 d_1}{d_1^2 \sqrt{k_4}} \right) \
              \left(  \frac{k_1^2}{k_2^2 d_1^2} \frac{1}{k_8 + C^2} \right)
              = p_5 \left( \frac{1}{ k_2^2 d_1^2 / k_1^2 (k_8+C^2)}\right)
              = \frac{p_5}{ p_6 + y_1^2}  \\
{\rm RHS(2)} &= \left( \frac{d_3}{d_1 \sqrt{k_4}} \right) K = \frac{d_3}{d_1} y_3 = p_7 y_2
\end{aligned}\end{gathered}
\end{equation*}


\section{Nondimensional model}
The resulting non-dimensional system of equations is
\begin{eqnarray}
\frac{dy_1}{dt} &=& \frac{1}{1 + p_1 y^2} + \frac{p_2}{1 + y_3^2} - y_1  \label{nondim_de_one} \\
\frac{dy_2}{dt} &=& \frac{a y_3^2}{1 + p_3 y^2} - p_4 y_2 \label{nondim_de_two} \\
\frac{dy_3}{dt} &=& \frac{r p_5}{p_6 + y_1^2} - p_7 y_3 \label{nondim_de_three}
\end{eqnarray}
subject to initial conditions
\begin{equation}
y_1(0) = y_{1,0}, \quad y_2(0) = y_{2,0}, \quad y_3(0) = y_{3,0}.
\end{equation}

\bibliographystyle{unsrt}
\bibliography{stem_cell}


\end{document}

