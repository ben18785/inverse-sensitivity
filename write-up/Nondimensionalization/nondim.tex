\documentclass[10pt,letterpaper]{article}
\usepackage[top=0.85in,left=2.75in,footskip=0.75in]{geometry}

% amsmath and amssymb packages, useful for mathematical formulas and symbols
\usepackage{amsmath,amssymb}
\DeclareMathOperator*{\argmax}{arg\,max}

\usepackage{graphicx}
\usepackage{booktabs}

% Use adjustwidth environment to exceed column width (see example table in text)
\usepackage{changepage}
\usepackage{tabularx}

% Use Unicode characters when possible
\usepackage{inputenc}

% textcomp package and marvosym package for additional characters
\usepackage{textcomp,marvosym}

% cite package, to clean up citations in the main text. Do not remove.
\usepackage{cite}

% Use nameref to cite supporting information files (see Supporting Information section for more info)
\usepackage{nameref,hyperref}

% line numbers
\usepackage[right]{lineno}

\usepackage{tikz}

% ligatures disabled
\usepackage{microtype}
\DisableLigatures[f]{encoding = *, family = * }


% array package and thick rules for tables
\usepackage{array}

\usepackage{algorithm}
\usepackage{algpseudocode}

\usepackage{float}
\usepackage{timestamp}
\usepackage{xfrac}
\usepackage{mathtools}

%% Include all macros below

\newcommand{\lorem}{{\bf LOREM}}
\newcommand{\ipsum}{{\bf IPSUM}}

%% END MACROS SECTION
\newcommand{\R}{\mathbb{R}}

\begin{document}
\vspace*{0.2in}

%% Title must be 250 characters or less.
%\begin{flushleft}
%{\Large
%\textbf\newline{A Monte Carlo method to estimate cell population heterogeneity}
%}
%\newline
%\\
%Ben Lambert\textsuperscript{1,2}*,
%David J. Gavaghan\textsuperscript{3},
%Simon Tavener\textsuperscript{4}.
%\\
%\bigskip
%\textbf{1} Department of Zoology, University of Oxford, Oxford, Oxfordshire, U.K.
%\\
%\textbf{2} MRC Centre for Global Infectious Disease Analysis, School of Public Health, Imperial College London, London W2 1PG, UK.
%\\
%\textbf{3} Department of Computer Science, University of Oxford, Oxford, U.K.
%\\
%\textbf{4} Department of Mathematics, Colorado State University, Fort Collins, Colorado, U.S.A.
%\\
%\bigskip
%
%% Use the asterisk to denote corresponding authorship and provide email address in note below.
%*ben.c.lambert@gmail.com.
%
%\end{flushleft}

\hfill Revision date \& time: \timestamp
\bigskip


% Please keep the abstract below 300 words
%%%%%%%%%%%%%%%%%%%%%%%%%%%%%%%%%%%%%%%%%%%%%%%%%%%%%%%%%%%%%%%%%%%%%%%%%%%%%%%%%%%%%%%%%%%%%%%%%%%%%%%%%%%%%%%%%%%%%%%%                                                                                                                  %                                                                                                                      %
%       ABSTRACT                                                                                                       %
%                                                                                                                      %
%%%%%%%%%%%%%%%%%%%%%%%%%%%%%%%%%%%%%%%%%%%%%%%%%%%%%%%%%%%%%%%%%%%%%%%%%%%%%%%%%%%%%%%%%%%%%%%%%%%%%%%%%%%%%%%%%%%%%%%%

% The Abstract of the paper should be succinct; it must not exceed 300 words. Authors should mention the techniques used without going into methodological detail and should summarize the most important results.

% While the Abstract is conceptually divided into three sections (Background, Methodology/Principal Findings, and Conclusions/Significance), do not apply these distinct headings to the Abstract within the article file.

% Do not include any citations. Avoid specialist abbreviations.

\section{Nondimensionalization}

\subsection{Initial value problem}

\begin{equation}
\begin{aligned}
\frac{dC}{d\tau} &= \frac{k_1}{k_2 + Z^2} + \frac{k_3}{k_4 + K^2} - d_1 C \\
\frac{dZ}{d\tau} &= \frac{a k_5 K^2}{k_6 + K^2} - d_2 Z \\
\frac{dK}{d\tau} &= \frac{r k_7}{k_8 + C^2} - d_3 K
\end{aligned}
\end{equation}

Initial conditions
\begin{equation}
C(0) = 1, \quad Z(0) = 0, \quad K(0) = 0
\end{equation}



\subsection{Units of dimensional parameters}

$C, K$ and $Z$ are concentrations with units $L^{-3}$ and therefore all the LHS terms e.g., $\frac{dC}{d\tau}$ have units $L^{-3}T^{-1}$

\begin{itemize}
  \item $d_1, d_2, d_3$ have units $T^{-1}$
  \item $k_2, k_4, k_6, k_8$ have units of $L^{-6}$ %so that the denominators have consistent units
  \item $k_1, k_3$ have units $L^{-9} T^{-1}$
  \item $k_7$ has units $L^{-9} T^{-1}$ if $r$ is dimensionless
  \item $k_5$ has units $L^{-3}T^{-1}$  if $a$ is dimensionless
\end{itemize}


\subsection{Nondimensional variables}
The nondimensional variables are ...
\begin{equation}
\begin{aligned}
y_1 = \frac{k_2 d_1}{k_1} C \\
y_2 = \frac{k_6 d_1}{k_4 k_5} Z \\
y_3 = \frac{1}{\sqrt{k_4}} K \\
\end{aligned}
\end{equation}

\begin{itemize}
  \item $\dfrac{k_2 d_1}{k_1}$ has units $L^3$ therefore $y_1$ is dimensionless
  \item $\dfrac{k_6 d_1}{k_4 k_5}$ has units $L^3$ therefore $y_2$ is dimensionless
  \item $\dfrac{1}{\sqrt{k_4}}$ has units $L^3$ therefore $y_3$ is dimensionless
\end{itemize}


\subsection{Nondimensional groups (parameters)}

The nondimensional groups (parameters) are ...
\begin{equation}
\begin{aligned}
p_1 &= \frac{k_4^2 k_5^2}{k_2 k_6^2 d_1} \\
p_2 &= \frac{k_2 k_3}{k_1 k_4} \\
p_3 &= \frac{k_4}{k_6} \\
p_4 &= \frac{d_2}{d_1} \\
p_5 &= \frac{k_2^2 k_7 d_1}{k_1^2 \sqrt{k_4}} \\
p_6 &= \frac{k_2^2 k_8 d_1^2}{k_1^2} \\
p_7 &= \frac{d_3}{d_1} 
\end{aligned}
\end{equation}

The following parameters are dimensionless trivially since they are ratios of dimensional parameter with equal units
\begin{itemize}
  \item $p_3$ is dimensionless
  \item $p_4$ is dimensionless
  \item $p_7$ is dimensionless
\end{itemize}
The remaining parameters are not so obviously dimensionless
\begin{itemize}
  \item $p_1$ has units $\dfrac{L^{-12}\cdot L^{-6}T^{-2}}               {L^{-6} \cdot L^{-12} \cdot T^{-2}}$ 
  \item $p_2$ has units $\dfrac{L^{-6} \cdot L^{-9}T^{-1}}               {L^{-9} T^{-1} \cdot L^{-6}}$
  \item $p_5$ has units $\dfrac{L^{-12}\cdot L^{-9}T^{-1} \cdot T^{-1} } {L^{-18} T^{-2} \cdot L^{-3}}$
  \item $p_6$ has units $\dfrac{L^{-12}\cdot L^{-6} \cdot T^{-2}}        {L^{-18}\cdot T^{-2}}$
\end{itemize}


\subsection{Nondimensionalizing governing equations}

The terms to multiply through by are most easily determined from the decay term.

\begin{itemize}
  \item Equation One -- multiply by $\dfrac{k_2}{k_1}$
  \item Equation Two -- multiply by $\dfrac{k_6}{k_4 k_5}$
  \item Equation Three -- multiply by $\dfrac{1}{d_1 \sqrt{k_4}}$
\end{itemize}

\subsubsection{Equation One}
\begin{equation}
\begin{aligned}
LHS &= \frac{1}{d_1} \frac{k_2 d_1}{k_1} \frac{dD}{d\tau} = \frac{1}{d_1} \frac{dy_1}{d\tau} \\
RHS(1) &= \frac{1}{1 + Z^2/k_2} = \frac{1}{1 + (k_4 k_5 y_2 / k_6 d_1)^2/k_2} = \frac{1}{1 + p_1 y_2^2} \\
RHS(2) &= \frac{k_2 k_3}{k_1 k_4} \frac{1}{1 + k_4 y_3^2/k_4} = \frac{p_2}{1 + y_3^2}  \\
RHS(3) &= \frac{k_2 d_1}{k_1} C = y_1
\end{aligned}
\end{equation}

\subsubsection{Equation Two}
\begin{equation}
\begin{aligned}
LHS &= \frac{1}{d_1} \frac{k_6 d_1}{k_4 k_5} \frac{dZ}{d\tau} = \frac{1}{d_1} \frac{dy_2}{d\tau} \\
RHS(1) &= \frac{k_6}{k_4} \frac{K^2}{k_6(1 + K^2/k_6)} = \frac{K^2/k_4}{(1 + k_4/k_6*K^2/k_4)} = \frac{y_3^2}{1 + p_3 y_3^2} \\
RHS(2) &= \frac{d_2}{d_1} * \frac{k_6 d_1}{k_4 k_5} Z = p_4 y_2
\end{aligned}
\end{equation}


\subsubsection{Equation Three}
\begin{equation}
\begin{aligned}
LHS &= \frac{1}{d_1 \sqrt{k_4}} \frac{dK}{d\tau} = \frac{1}{d_1} \frac{dy_3}{d\tau} \\
RHS(1) &= \frac{k_2^2 k_7 d_1}{d_1^2 \sqrt{k_4}} * \left(  \frac{k_1^2}{k_2^2 d_1^2} \frac{1}{k_8 + C^2} \right)
  = p_5 \left( \frac{1}{ k_2^2 d_1^2 / k_1^2 (k_8+C^2)}\right) 
  = p_5 \frac{1}{ p_6 + y_1^2}  \\
RHS(2) &= \frac{d_3}{d_1 \sqrt{k_4}}K = \frac{d_3}{d_1} y_3 = p_7 y_2
\end{aligned}
\end{equation}




\end{document}

