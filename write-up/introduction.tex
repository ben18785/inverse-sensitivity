\section{Introduction}
Variation, as opposed to homogeneity, is the rule rather than exception in biology. Indeed, without variation, biology as a discipline would not exist, since as evolutionary biologist JBS Haldane wrote, variation is the ``raw material" of evolution. The Red Queen Hypothesis asserts organisms must continually evolve in order to survive when pitted against other - also evolving - organisms \cite{ridley1994red}. A corollary of this hypothesis is that multicellular organisms should evolve cellular phenotypic heterogeneity to allow faster adaptation to changing environments, which may explain the observed variation in a range of biological systems \cite{fraser2009chance}. Whilst cell population variation can confer evolutionary advantages, it can also be costly in other circumstances. In biotechnological processes, heterogeneity in cellular function can lead to reduced yields of biochemical products \cite{delvigne2014metabolic}. In human biology, variation across cells can enable pathologies to develop and also prevents effective treatment, since medical interventions typically aim to steer modal cellular properties and hence fail to influence key subpopulations. For example, cellular heterogeneity helps some cancerous tumours to persist \cite{gatenby2007cellular} and can make tumours more likely to evolve resistance to chemotherapies \cite{altrock2015mathematics}. Identifying and quantifying sources of variation in populations of cells is important for wide ranging applications to discern whether the variability is benign or alternatively requires remedy.

Mathematical models are essential tools for understanding cellular systems, whose emergent properties are the result of complex interactions between various actors. Perhaps the simplest flavour of mathematical model used in biological systems is an ordinary differential equation (ODE) that lumps individual actors into partitions according to structure or function, and seeks to model the mean behaviour of each compartment. Data from population-averaged experimental assays can determine whether such models faithfully reproduce system behaviours and can be used to understand the interactions of various cellular components of complex metabolic, signalling and transcriptional networks. The worth of such models however depends on whether averages mask substantial differences in individual behaviour \cite{altschuler2010cellular}. In some cases, differences in cellular protein abundances due to biochemical ``noise" are not biologically meaningful  \cite{elowitz2002stochastic} and the system is well described by average cell behaviour. In others there are functional consequences. For example, a laboratory study demonstrated that subpopulations of clonally-derived hematopoietic progenitor cells with either low or high expression of a particular stem cell marker resulted in different blood lineages \cite{chang2008transcriptome}.

To accommodate cell population heterogeneity in mathematical models, many modelling frameworks are available, each posing different challenges for parameter inference. A recent review is presented in \cite{waldherr2018estimation}. These approaches include modelling biochemical processes stochastically using reaction-diffusion equations (RDEs), where properties of ensembles of cells are represented by probability distributions which evolve according to chemical master equations. See \cite{erban2007practical} for a tutorial on RDEs. Alternatively, population balance equations (PBEs) can be used, which determine the dynamics of the ``number density" of differing cell types. In PBEs, cell properties are represented as points in $\mathbb{R}^n$ which, in turn, affect their function, including their rate of death and cell division. In a PBE approach, variation in measured quantities results primarily from differing functional properties of heterogeneous cell types and variable initial densities of each type. See \cite{ramkrishna2014population} for an introduction to PBEs.

Here, we suppose heterogeneity in quantities of interest across cells is generated by idiosyncratic variation in the rates of cellular processes. The modelling approach we follow is similar to that of \cite{dixit2018maximum} and is based on an ODE framework. In our model, each cell evolves according to an ODE, with its progression directed by parameters whose value varies between cells. To our knowledge, this flavour of model is unnamed and so, for sake of reference, we call them ``heterogenous ODE" models (HODEs). In HODEs, the aim of inference is to estimate distributions of parameter values across cells consistent with observations. A benefit of using HODEs is that these models are computationally straightforward to simulate and, arguably, simpler to parameterise than PBEs. By using HODEs, we assume that most observed variation comes from differences in biological processes across cells, not inherent stochasticity in biochemical reactions within cells as in stochastic RDEs.

Inference for HODEs is problematic due partly to the experimental hurdles involved with generating data of sufficient standard. Unlike models which represent a population by a single scalar ODE, since HODEs are individual-based, they ideally require individual cell data for estimation. A widely-used method for generating such data is flow cytometry, where a large number of cells are streamed individually through a laser beam and, for example, the concentrations of fluorescently-labelled proteins are measured \cite{telford2012flow}. Other experimental techniques, including Western blotting and cytometric fluorescence microscopy, can also generate single cell measurements \cite{hughes2014single,hasenauer2011identification}. These experimental methods are all however destructive, meaning individual cells are sacrificed during measurement, and  observations at each time point hence represent \emph{``snapshots"} of the underlying population \cite{hasenauer2011identification}. These snapshots can be described by histograms \cite{dixit2018maximum} or density functions \cite{waldherr2018estimation} fit to measurements of quantities of interest. Since HODEs assume the state of each cell evolves continuously over time, experimental data tracing individual cell trajectories through time constitutes a richer data resource. The demands of obtaining this data are however higher and typically involve either tracking individual cells through imaging methods \cite{hilsenbeck2016software}, or trapping cells in a spatial position where they can be monitored over time \cite{fritzsch2012single}. These techniques however impose severe restrictions on experimental practices and cannot be used in many circumstances, including for online monitoring of biotechnological processes or analysis of \textit{in vivo} studies. For this reason, ``snapshot'' data continues to play an important role for determining cell level variability in many applications.

A variety of approaches have been proposed to estimate cellular variability by fitting HODES to snapshot data. In HODEs, parameter values vary across cells according to a to-be-determined probability distribution, and the solution to the inverse problem requires solving the cell-specific ODE system many times for each individual. The count of cells in experiments typically exceeds $\sim10^4$ \cite{hasenauer2011identification}, and so approaches where the computational burden scales with this count are usually infeasible. Instead, approaches that approximate raw snapshot data by fitting probability densities to them are typically more efficient and commonly used \cite{hasenauer2011identification,hasenauer2014ode,loos2018hierarchical,dixit2018maximum}, and we follow this approach here. We now briefly describe the existing approaches for using HODE models to estimate cell population heterogeneity. Hasenauer et al. (2011) present a Bayesian approach to inference for HODEs, which models the input parameter space using an ansatz of a mixture of densities of chosen types. The authors then use their method to reproduce population substructure on synthetic data generated from a model of tumour necrosis factor stimulus. Hasenauer et al. (2014) use mixture models to model subpopulation structure in snapshot data with multiple-start local optimisation employed to maximise the non-convex likelihood, which they then apply to synthetic and real data from signalling pathway models. Loos et al. (2018) also use mixture models to represent subpopulation structure and use maximum likelihood to estimate both within- and between-subpopulation variability, which permits fitting to multivariate output distributions with complex correlation structures. Dixit et al. (2018) assign observations into discrete bins, then choose likelihood distributions using the principle of maximum entropy to estimate cell variability within a Bayesian framework.

Our framework is Bayesian although is distinct from the approach used to fit many dynamic models, since we assume output variation arises from parameter heterogeneity across cells, with no contribution from measurement noise. The approach is hence most suitable when measurement error is minimal. Our method is a two-step Monte Carlo approach which, for reasons described in \S \ref{sec:method}, we call ``Contour Monte Carlo'' (CMC). Unlike many existing methods, CMC is straightforward to implement and does not require extensive computation time. CMC uses MCMC in its second step to sample from the posterior distribution meaning it does not require ansatz densities. It also does not require the number of cell clusters be chosen beforehand, rather, subpopulations emerge as modes in the posterior parameter distributions. Like \cite{loos2018hierarchical}, CMC can fit multivariate snapshot data and unlike \cite{dixit2018maximum}, does not use discrete bins to model continuous data. As more experimental techniques elucidating single cell behaviour are developed, interest in models describing measurement snapshots should follow. We argue that due to its simplicity and generality, CMC can be used to perform inference on the proliferation of rich single cell data, and thus, is a useful addition to the modeller's toolkit.


\underline{Outline of the paper}: In \S \ref{sec:method}, we describe our probabilistic model of the inverse problem and detail the CMC algorithm for generating samples from the posterior parameter distribution. In \S \ref{sec:results}, we use CMC to estimate cell population heterogeneity in three systems of biological interest.


